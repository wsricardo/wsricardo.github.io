\documentclass[12pt,a4paper]{article}

% ==== Pacotes de idioma e codificação ====
\usepackage[utf8]{inputenc}   % Permite acentos diretamente no código
\usepackage[T1]{fontenc}      % Fonte com acentos corretos no PDF
\usepackage[brazil]{babel}    % Hifenização e termos em português

% ==== Matemática ====
\usepackage{amsmath, amssymb, amsfonts}
\usepackage{mathtools}

% ==== Layout ====
\usepackage[margin=3.8cm]{geometry}
\usepackage{enumitem}         % Controle de listas
\usepackage{fancyhdr}         % Cabeçalho e rodapé
\usepackage{setspace}         % Espaçamento entre linhas

% ==== Aparência ====
\usepackage{lmodern}          % Fonte Latin Modern
\usepackage{microtype}        % Melhora justificação
\usepackage{titlesec}         % Personalizar títulos

% ==== Configuração de cabeçalho/rodapé ====
\pagestyle{fancy}
\fancyhf{}
\lhead{Cálculo I}
\rhead{Lista de Exercícios}
\cfoot{\thepage}

% ==== Estilo dos títulos ====
\titleformat{\section}{\normalfont\Large\bfseries}{\thesection}{1em}{}

% ==== Início do documento ====
\begin{document}
	
	% ==== CAPA ====
	\begin{center}
		\textbf{\LARGE Lista de Exercícios — Cálculo I} \\[0.5cm]
		\textbf{Tema: Limites, Derivadas e Integrais} \\[1cm]
	\end{center}
	
	\noindent
	\textbf{Nome:} \rule{10cm}{0.4pt} \hfill
	
	\noindent
	\textbf{Data:} \rule{4cm}{0.4pt} \\[0.4cm]
	
	\noindent
	\textbf{Instruções:} Resolva cada questão detalhando o raciocínio.  
	Use justificativas matemáticas claras, mostre todos os passos.  
	Questões podem exigir uso de séries, substituições e técnicas avançadas.  
	
	\vspace{0.8cm}
	
	% ==== LISTA ====
	\begin{enumerate}[leftmargin=0.8cm,label=\textbf{Questão \arabic*},itemsep=1.6cm]
		
		% ===== LIMITES (2) =====
		\item Calcule:
		\[
		\lim_{x\to 0}\frac{\sin x - x + \dfrac{x^3}{6}}{x^5}.
		\]
		
		\item Calcule:
		\[
		\lim_{x\to 0}\frac{\ln(1+x) - x + \dfrac{x^2}{2}}{x^3}.
		\]
		
		% ===== DERIVADAS (4) =====
		\item Seja
		\[
		f(x)=\frac{x^{\sin x}\,e^{x^2}}{\sqrt{1+x^4}}.
		\]
		Calcule \(f'(x)\) explicitamente usando diferenciação logarítmica.
		
		\item Seja \(g(x)=\ln\big(\sin(x^2)\big)\) para \(x\) tal que \(\sin(x^2)>0\).  
		Calcule \(g'(x)\) e \(g''(x)\).
		
		\item A curva é dada por
		\[
		x^2 y + e^{xy} = \sin y + 1.
		\]
		Determine \(\dfrac{dy}{dx}\) em termos de \(x\) e \(y\).
		
		\item Encontre e classifique todos os extremos locais e globais de
		\[
		h(x)=x e^{-x^2/2},\quad x\in\mathbb{R}.
		\]
		
		% ===== INTEGRAIS (4) =====
		\item Calcule:
		\[
		\int_{0}^{\infty} x^3 e^{-x^2}\,dx.
		\]
		
		\item Calcule:
		\[
		\int_{0}^{2\pi}\frac{d\theta}{5-4\cos\theta}.
		\]
		
		
		\item Calcule:
		\[
		\int_{0}^{1}\frac{\ln(1+x)}{x}\,dx.
		\]
		
		
		\item Calcule:
		\[
		\int \frac{x^2}{\sqrt{x^2+1}}\,dx.
		\]
		
	\end{enumerate}
	
	
	\vfill
	\begin{center}
		\textit{Bons estudos! Lembre-se: o caminho é tão importante quanto a resposta final.}
	\end{center}
	
\end{document}
